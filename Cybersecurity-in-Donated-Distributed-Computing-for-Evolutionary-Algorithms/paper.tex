\documentclass[graybox]{svmult}

\usepackage{type1cm}
\usepackage{makeidx}
\usepackage{graphicx}
\usepackage{multicol}
\usepackage[bottom]{footmisc}
\usepackage{newtxtext}
\usepackage{newtxmath}

\makeindex

\begin{document}

\title*{Cybersecurity in Donated Distributed Computing for Evolutionary Algorithms}
\titlerunning{Cybersecurity in DDC for EA}

\author{Petar Tomov, Iliyan Zankinski, Todor Balabanov \textsuperscript{\tiny{0000-0003-3139-069X}}}
\authorrunning{P. Tomov et al.}
\institute{Petar Tomov \at Institute of Information and Communication Technologies - Bulgarian Academy of Sciences, acad. Georgi Bonchev Str., block 2, 1113 Sofia, Bulgaria \email{p.tomov@iit.bas.bg}
\and Iliyan Zankinski \at Institute of Information and Communication Technologies - Bulgarian Academy of Sciences, acad. Georgi Bonchev Str., block 2, 1113 Sofia, Bulgaria \email{iliyan@hsi.iccs.bas.bg}
\and Todor Balabanov \at Institute of Information and Communication Technologies - Bulgarian Academy of Sciences, acad. Georgi Bonchev Str., block 2, 1113 Sofia, Bulgaria \email{todorb@iinf.bas.bg}}
%
% Bulgarian Academy of Sciences
% Institute of Information and Communication Technologies
% acad. Georgi Bonchev Str., block 2, office 514, 1113 Sofia, Bulgaria
% http://www.iict.bas.bg/
% iict@bas.bg
%

\maketitle

\abstract*{Donated distributed computing, also known as volunteer computing, is a form of distributed computing that is organized as a public donation of calculating resources. Donated calculating power can involve thousands of separate CPUs and it can achieve the performance of a supercomputer. In most of the cases donated distributed computing is organized by open source software, which can lead to the involvement of many more volunteers. This research focuses on cybersecurity issues when donated distributed computing is used for optimization with evolutionary algorithms.}

\abstract{Donated distributed computing, also known as volunteer computing, is a form of distributed computing that is organized as a public donation of calculating resources. Donated calculating power can involve thousands of separate CPUs and it can achieve the performance of a supercomputer. In most of the cases donated distributed computing is organized by open source software, which can lead to the involvement of many more volunteers. This research focuses on cybersecurity issues when donated distributed computing is used for optimization with evolutionary algorithms.}

\section{Introduction}
\label{sec:01}

A subset of calculating problems has the characteristic that different computation steps are not strongly connected to each other and those steps can be calculated separately. In such problems, parallel programming is used and computation time is usually reduced by involving more calculating units. According to who owns and administers the calculating resources there are two groups of parallel computing - calculating devices under your own control and calculating devices under somebody else control. Part of the first group (private owned) is single machines with multi-cores or multiprocessors, supercomputer with many identical modules, a grid of identical single machines and a cluster of different single machines. It does not matter what is the hardware specification in the first group, which is common for the first group is that the organizer of the parallel computations has full access and full control over the hardware. This situation is very important when cybersecurity in massive computing is discussed. In the second group different type of computing hardware is organized to compute a common task, but calculating devices are not under the control of the person or the organization who/which is doing the computations. When the involved devices are participating in a volunteer principle there is no control of any kind over the donated calculating power. The biggest advantage of the donated distributed computing is that it comes at a very low price. The biggest disadvantage of the donated distributed computing is that it comes with a wide list of security and reliability issues. 

General cybersecurity does not differ in donated distributed computing than the other network communication software solutions. What is more, in donated distributed computing is related to calculation accuracy and reliability of the results. Numerical calculations that are done on different hardware can lead to different results just because of the machine word, floating-point numbers presentation and different error handling. Such differences are not human provoked and are directly math dependent. A bigger problem are human interventions with calculation instructions executed on the device and the information exchanged with the central distributed computing infrastructure. Even that it is relatively complicated a malicious user is capable to change what is calculated on his/her device directly in the device memory. A malicious user is capable to change the input data and to manipulate the output data. The general problem for the organizer of the donated distributed computing infrastructure is that he/she can not trust any of the volunteer participants. 

This study is devoted to cybersecurity issues in donated distributed computing when it is used for evolutionary algorithms. After the introductory part, the paper is organized as follows: Evolutionary algorithms in donated distributed computing are discussed; After that, some examples of damaged reliability are presented; And finally, conclusions and some further research are proposed. 

\section{Evolutionary Algorithms}
\label{sec:02}

\section{Conclusions}
\label{sec:03}

\begin{acknowledgement}
This research is funded by Velbazhd Software LLC and it is partially supported by the Bulgarian Ministry of Education and Science (contract D01–205/23.11.2018) under the National Scientific Program ``Information and Communication Technologies for a Single Digital Market in Science, Education and Security (ICTinSES)'', approved by DCM \# 577/17.08.2018.
\end{acknowledgement}

\begin{thebibliography}{99.}

\bibitem{g-stock-01} Chokesatean, P.: Credibility-based Binary Feedback Model for Grid Resource Planning. University of Pittsburgh (2008)

\bibitem{mql5-cloud-network-01} Chang, C., Narayana Srirama S., Buyya, R.: Indie Fog An Efficient Fog-Computing Infrastructure for the Internet of Things. Computer \textbf{50}(9), 92--98 (2017)

\bibitem{money-bee-01} Bohn, A., Guting, T., Mansmann, T.: MoneyBee Aktienkursprognose mit kunstlicher intelligenz bei hoher rechenleistung. Wirtschaftsinf \textbf{45}, 325--333 (2003)

\bibitem{genetic-algorithms-01} Donate, J.P., Li, X., Sanchez, G.G., Miguel, A.S.: Time series forecasting by evolving artificial neural networks with genetic algorithms, differential evolution and estimation of distribution algorithm. Neural Computing and Applications \textbf{22}, 11--20 (2013)

\bibitem{time-series-01} Bandt, C., Pompe, B.: Permutation Entropy A Natural Complexity Measure for Time Series. Physical Review Letters, \textbf{88}(17), 174102 (2002)

\bibitem{time-series-02} Salvador, S., Chan, P.: Toward Accurate Dynamic Time Warping in Linear Time and Space. Intelligent Data Analysis, \textbf{11}(5) 561--580 (2007)

\bibitem{time-series-03} Lan, Y.: Computational Approaches for Time Series Analysis and Prediction Data-Driven Methods for Pseudo-Periodical Sequences. University of Bradford (2009)

\bibitem{vitosha-trade-01} Zankinski, I., Barova, M., Tomov P.: Hybrid Approach Based on Combination of Backpropagation and Evolutionary Algorithms for Artificial Neural Networks Training by Using Mobile Devices in Distributed Computing Environment. Proceedings of Large-Scale Scientific Computing, 425--432 (2018)

\bibitem{distributed-computing-01} Megiddo, N.: Applying parallel computation algorithms in the design of serial algorithms. Journal of the Association for Computing Machinery \textbf{30}(4), 852--865 (1983)

\bibitem{crowdsensing-01} Leppanen, T., Alvarez Lacasia, J., Tobe, Y., Sezaki, K., Riekki, J.: Mobile crowdsensing with mobile agents. Autonomous Agents and Multi-Agent Systems \textbf{31}, 1--35 (2017)

\bibitem{voting-01} Laukkanen, S., Kangas, A., Kangas, J.: Applying voting theory in natural resource management A case of multiple-criteria group decision support, Journal of Environmental Management, \textbf{64}(2), 127--137 (2002)

\end{thebibliography}


\end{document}
